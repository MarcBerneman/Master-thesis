This work started with a summary of basic and advanced methods for estimating nonparametric models of LTI systems. Then, a review of data-driven model reference control was given. The cost functions were formulated in the time domain. However, by translating these formulations into the frequency domain it becomes clear that nonparametric models are hidden in the maths.

We then proposed to combine advanced methods for estimating nonparametric models with data-driven model reference control. A weighted nonlinear least squares cost function was also proposed. The time domain, frequency domain and weighted nonlinear least squares methods were applied to discrete-time systems. The weighted nonlinear least squares method greatly improves the quality of the controller in the case that the ideal controller is realizable. However, in most cases the original time domain method works better than the others.

The advantage of the proposed frequency domain method is that it is more general as it can also be applied to continuous-time systems. A nonparametric model of a continuous-time system was estimated and this was used to design an analog controller. Moreover, a review was given of constraints that guarantee the stability of the closed loop system. These constraints were used to verify that the designed controller would not destabilize the system.

\paragraph{Future work}
Model reference control is useful for finding a controller when input-output data of the uncontrolled system are available. This input-output data can be used to find a nonparametric model of the frequency response function of the system, which can then be used to find a suitable controller. However, it is still useful to have a parametric model of the system for the sake of interpretation.

An area where model reference control could be beneficial is in the control of time-varying systems. A linear parametric model can be estimated for the system, which can be used to design a controller. Afterwards, the controller can be adapted over time by using model reference control. This way, the interpretability of parametric models and the simplicity of nonparametric models can be combined into an adaptive control scheme. Employing nonparametric estimation of time-varying systems might also prove to be useful in this regard \cite{Lataire_time-varying}.


\paragraph{Contributions}
This work started when my supervisors pointed me to a paper comparing model-based and data-driven control \cite{comparison_model-based_data-driven}. In that paper, the correlation-based approach was mentioned. This led me to a paper on model reference control \cite{Data-driven_model_reference_control}. I was able to recreate the results presented in that paper. Afterwards, while trying to understand the correlation-based approach in depth, I noticed that nonparametric models of the frequency response function of the uncontrolled system were hiding in the mathematics. This then led to the use of more advanced nonparametric frequency domain methods. Finally, I realized that working in the frequency domain also allows for a generalization of the methods to continuous-time systems.