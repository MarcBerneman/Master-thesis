\selectlanguage{english}
\masterTitle{Title}{A frequency domain approach to data-driven control}
\authorThesis{Author}%in this input frame you write "Author" in the language of the abstract
\Master{Master of Science in Electrical Engineering -- major in Measuring, Modeling and Control}% here you have to write the full title of your Master programme
\yearTitle{Academic year}
%% in this environment, you will write your abstract. Keep it limited to a SINGLE PAGE. Max. 500 words should normally do the trick
\begin{abstract}
	\section*{Abstract}\label{sec:abstract}
% 	\addcontentsline{toc}{section}{\nameref{sec:abstract}}
	%% start adjusting the abstract from here on
	{\fontsize{10}{16}
	%\setstretch{1.05}
    
Data-driven model reference control allows for the design of a controller from input and output data when a parametric model of the system is not available. It is already known how to do this for discrete-time systems. 
%By using nonparametric models of the frequency response function of linear time-invariant systems it is possible to generalize model reference control to continuous-time systems. 
In this thesis we propose to generalize model reference control to continuous-time systems.
Moreover, it is demonstrated how the stability of the closed loop system can be guaranteed by using the small-gain theorem. The proposed methods are used to design an analog controller for a continuous-time system.

	\vspace{5mm}
	\noindent\textbf{Keywords:} data-driven controller tuning; model reference control; frequency domain approach; nonparametric; frequency response function
	}
    
\end{abstract}
\newpage