\section{Miscellaneous}
\subsection{Inner working of lsim}
The function \emph{lsim} can simulate continuous-time systems. However, the input is given in discrete time. So care must be taken when using this function. The simulation can occur by using either a ZOH or a FOH. If you are simulating a CT system with a ZOH, you are actually measuring the ZOH DT version and not the CT version of the TF. In practice, the ZOH is followed by a reconstruction filter that compensates for this. In simulation you can get around this by using a higher sampling frequency. This works because
\begin{equation*}
    G(j 2 \pi f) \approx G_{ZOH}(e^{j 2 \pi f T_s})
\end{equation*}
when $f << F_s/2$.


\subsection{Put this elsewhere}
It is then possible to construct $J_{T,l_1}$ by taking the inverse discrete Fourier transform (IDFT) of $\Phi_{u_W \epsilon}$.\todo{Put this part somewhere else maybe}
\begin{equation*}
    R_{u_W \epsilon}(\tau,\rho) = \frac{1}{T} IDFT\{\Phi_{u_W \epsilon}(\Omega_k,\rho)\}
\end{equation*}
\todo[inline]{write about parceval theorem and how sum in FD is equivalent to sum in TD. by taking subset 2l1+1 in TD you get a less accurate measure of sum in FD, but with less bias as a result}